\RequirePackage{plautopatch}  % pLaTeX または upLaTeX のとき
%\documentclass[uplatex,dvipdfmx,titlepage,a4j]{jsarticle}% upLaTeX のとき
\documentclass[dvipdfmx,titlepage,a4j]{jsarticle}  % pLaTeX のとき
\usepackage{listings,jvlisting}
\usepackage{amsmath,amssymb}
\usepackage{graphicx}
\usepackage[yen]{okuverb}
\usepackage{r04ec-exp}
\usepackage{here}
\usepackage{ascmac}
\usepackage{fancybox}
\usepackage{fancyvrb}
\usepackage{fancyhdr}
\usepackage{lastpage}

\fancypagestyle{foot}
{
\fancyhead[C]{信号処理プログラミング}
\fancyfoot[C]{\thepage / \pageref{LastPage}}
\renewcommand\headrulewidth{0.4pt}
}

%ここからソースコードの表示に関する設定
\lstset{
  language={C++},
  basicstyle={\ttfamily},
  identifierstyle={\small},
  commentstyle={\smallitshape},
  keywordstyle={\small\bfseries},
  ndkeywordstyle={\small},
  stringstyle={\small\ttfamily},
  frame={tb},
  tabsize={2},
  breaklines=true,
  columns=[l]{fullflexible},
  numbers=left,
  xrightmargin=0zw,
  xleftmargin=3zw,
  numberstyle={\scriptsize},
  stepnumber=1,
  numbersep=1zw,
  lineskip=-0.5ex
}

\renewcommand{\lstlistingname}{リスト}
%ここまでソースコードの表示に関する設定

\title{フーリエ変換について}
% 学年・番号
\grade{4年42番}%
% 氏名
\author{鷲尾 優作}
% 班(後期は班に分かれて実験をする.そのときは,ここに班番号を記入する.)
\team{}
% 提出日
\date{2022年10月3日}
% 実験日
\expdate{}
% 共同実験者
% グループに分かれて実験をするテーマでは,グループメンバーの番号名前を書く.
\coauthor{}
%
%記載例:
%\coauthor{%
%  2番 & 新潟 花子\\
%  11番 & 三条 次郎}
%%

\begin{document}
\pagestyle{foot}

\maketitle

\section{フーリエ級数展開}

\paragraph{Q1 (1)式中の基本振動を表す項を書きだせ.また,5倍振動を表す項を書き出せ.}
\paragraph{Q2 (1)式中の振動項の時間変化はいくらか.また,$b_0/2$の値の意味を答えよ.}
\paragraph{Q3 (1)式に含まれる振動項の振幅を求めよ.また,その振動をcos($n\omega_0t$)と比較した時の位相のズレを求めよ.}
\paragraph{Q4 (1)から(3)に書き換える場合,係数$c_n$はどのように決められるか.$a_n$,$b_n$との関係式を求めよ.}
\paragraph{Q5 (3)式の中で,3倍振動をを表す項を書き出せ.}
\paragraph{Q6 $c_n$と${c_\_}_n$の関係を答えよ.ただし,$f(t)$は実関数とする.}
\paragraph{Q7 n倍振動項(n>0)について考える.複素数$c_n$を極形式で,$c = \left\lvert c_n \right\rvert exp(i\delta_n)$
  のように書いたとする.絶対値$\left\lvert c_n \right\rvert$と偏角$\delta_n$は,何を表しているか.(Q3と比較せよ.)}
\paragraph{Q8 (4)式が成り立つことを確かめよ.}
\paragraph{Q9 (5)式が成り立つことを確かめよ.}

\section{フーリエ級数からフーリエ変換へ}
\paragraph{Q10 フーリエ成分$X(\omega)$は何を意味しているのだろうか.}

\section{フーリエ変換の実際の問題への適用: DFT}
\paragraph{Q11 (16)式から,$X_l$=$X_{l+N}$を示せ.}
\paragraph{Q12 $X_l$=$X_{l+N}$ということは,フーリエ成分$X_l$と$X_{l+N}$とが区別できないことを示している.
  これは扱うデータが離散的だからである.この事情を図を用いて説明せよ.}
\paragraph{Q13 $X_{N-2}$と$X_2$の間にはどんな関係があるか.}
\paragraph{Q14 フーリエ成分$\{X_l\}$から,元の信号列$\{X_k\}$を再現するための式を作れ.(DFTの逆変換)}

\begin{thebibliography}{99}
  \bibitem{ataka} 高橋 章、実験テキスト「信号処理プログラミング」、(2022年),
  \bibitem{ataka} 高橋 章、Wikipedia「標本化定理」、{https://ja.wikipedia.org/wiki/標本化定理} 、(2021年12月15日 (水) 12:16)
\end{thebibliography}

\end{document}